\chapter{字符串}

\section{KMP}

\paragraph{前缀函数}

对于长为 $n$ 的字符串 $s$,定义每个位置的前缀函数 $\pi(i)$,值为右端在 $i$ 的相等真后缀与真前缀中最长的长度。

设最长的长度为 $j_1=\pi(i)$,如何找到其次长 $j_2$?

注意到后缀 $j_1$ 位与前缀 $j_1$ 位完全相同,故 $j_2$ 为前缀 $j_1$ 中相等真前缀与真后缀中最长的,即
\[j_{n+1} = \pi(j_n-1)\]

\lstinputlisting[style=cpp,caption=/字符串/KMP/01.cpp]{字符串/KMP/01.cpp}

\paragraph{Knuth - Morris - Pratt}

给定一个文本 $t$ 和一个字符串 $s$ (模式串),尝试找到 $s$ 在 $t$ 中所有出现。

构造字符串 $s+*+t$,其中 $*$ 为不出现在两个字符串中的特殊字符,此时字符串 $t$ 的前缀恰为 $s$,$\pi(i)$ 的意义为 $s$ 在此处的出现长度。

当 $\pi(i)=|s|$ 时,$s$ 在此处完全出现。

当字符串已经合并时,直接计算 $\pi(i)$ 函数即可,字符串出现位置是 $i-2|s|$。

\lstinputlisting[style=cpp,caption=/字符串/KMP/02.cpp]{字符串/KMP/02.cpp}


