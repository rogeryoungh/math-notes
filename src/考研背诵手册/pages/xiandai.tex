\chapter{线性代数}

\newcommand{\transpose}[1]{{#1}^\mathsf{T}}

\section{行列式}

\subsection{大纲要求}

1. 了解行列式的概念,掌握行列式的性质。

2. 会应用行列式的性质和行列式按行(列)展开定理计算行列式。

\subsection{余子式}

在 $n$ 阶行列式中,把 $a_{ij}$ 元列所在的行列划去后
\begin{itemize}
	\item 余子式 $M_{ij}$:留下来的 $n-1$ 阶行列式;
	\item 代数余子式 $A_{ij} = (-1)^{i+j} M_{ij}$;
\end{itemize}

\begin{theorem}
	对于 $n$ 阶行列式 $|A|$ 有
	\[|A| = \sum_{j=1}^na_{ij}A_{ij} = \sum_{i=1}^na_{ij}A_{ij}\]
	前者称为 $n$ 阶行列式按第 $i$ 行的展开式,后者称为按第 $j$ 列的展开式。
\end{theorem}

若对不同行列展开,比如对于 $n$ 阶行列式 $|A|$ 的第 $k, i$ 行展开($k \neq i$)
\[\sum_{j=1}^na_{ij}A_{kj} = 0 \]

\section{矩阵}

\subsection{大纲要求}

1. 理解矩阵的概念,了解单位矩阵、数量矩阵、对角矩阵、三角矩阵、对称矩阵和反对称矩阵以及它们的性质。

2. 掌握矩阵的线性运算、乘法、转置以及它们的运算规律,了解方阵的幂与方阵乘积的行列式的性质。

3. 理解逆矩阵的概念,掌握逆矩阵的性质以及矩阵可逆的充分必要条件,理解伴随矩阵的概念,会用伴随矩阵求逆矩阵。

4. 理解矩阵初等变换的概念,了解初等矩阵的性质和矩阵等价的概念,理解矩阵的秩的概念,掌握用初等变换求矩阵的秩和逆矩阵的方法。

5. 了解分块矩阵及其运算。

\subsection{矩阵}

由 $sn$ 个数排成的 $s$ 行(横的)$n$ 列(纵的)表
\[ \left(
	\begin{matrix}
			a_{11} & a_{12} & \ldots & a_{1n} \\
			a_{21} & a_{22} & \ldots & a_{2n} \\
			\vdots & \vdots &        & \vdots \\a_{s1}&a_{s2}&\ldots&a_{sn}\\
		\end{matrix}
	\right) \]
称为一个 $s\times n$ 矩阵,记作 $A_{s\times n}$ 或 $A=(a_{ij})$,它的 $(i,j)$ 元也记作 $A(i;j)$。

有名字的矩阵
\begin{itemize}
	\item 零矩阵:全 $0$ 的矩阵,记作 $O$;
	\item 对角矩阵:只有对角线上的元素非 $0$ 的矩阵;
	\item 单位矩阵:主对角线上的元素为 $1$ 的矩阵,记作 $E$ 或 $I$;
	\item 数量矩阵:主对角线上的元素均为 $k$ 的对角矩阵,等于 $kE$;
	\item 上三角矩阵:主对角线及以上的元素非零的矩阵;
	\item 下三角矩阵:主对角线及以下的元素非零的矩阵;
	\item 对称矩阵:满足 $\transpose{A} = A$ 的矩阵;
	\item 反对称矩阵:满足 $\transpose{A} = -A$ 的矩阵;
	\item 正交矩阵:满足 $\transpose{A}A=E$ 的矩阵。
\end{itemize}

\subsection{矩阵的初等变换}

下面三种变换记作矩阵的初等行变换:
\begin{itemize}
	\item $r_i \leftrightarrow r_j$:对换 $i, j$ 两行;
	\item $r_i \times k$:第 $i$ 行乘数 $k$;
	\item $r_i + kr_j$:把第 $j$ 行的 $k$ 倍加到 $r_i$ 上;
\end{itemize}
把行换成列(把记号 $r$ 换成 $c$)即是初等列变换。统称初等变换。

可以用初等变换把矩阵变为行阶梯型矩阵:
\begin{itemize}
	\item 非零行在零行之上;
	\item 每行首个非零元的列坐标递增;
\end{itemize}
进一步的,再满足条件
\begin{itemize}
	\item 首个非零元为 $1$;
	\item 非零元所在列其他元为 $0$;
\end{itemize}
则称为行最简形矩阵。

矩阵等价:如果矩阵 $A$ 能经过有限次初等变换变为 $B$,则称 $A$ 与 $B$ 等价,记 $A \sim B$。若仅限行变换或者列变换,则称行(列)等价。

求矩阵的逆:对矩阵 $(A \mid E)$ 做初等行变换,当 $A$ 变为 $E$,$E$ 则变为 $A^{-1}$。

\subsection{矩阵的性质}

伴随矩阵:设矩阵 $A = (a_{ij})$,那么 $A$ 的伴随矩阵为
\[ A^*=(A_{ij}) \]
即每行每列全为代数余子式。

矩阵的逆、伴随、转置运算可交换
\[ \transpose{\left(A^{-1}\right)} = \left(\transpose{A}\right)^{-1}, \quad \transpose{\left(A^{*}\right)} = \left(\transpose{A}\right)^{*}, \quad \left(A^{*}\right)^{-1} = \left(A^{-1}\right)^{*} \]
反向性质
\[ \left(AB\right)^{-1} = B^{-1}A^{-1}, \quad \transpose{\left(AB\right)} = \transpose{B} \transpose{A}, \quad \left(AB\right)^{*} = B^{*}A^{*} \]
行列式
\[ |A^{-1}| = |A|^{-1}, \quad |\transpose{A}| = |A|, \quad |A^{*}| = |A|^{n-2} \quad (n \geqslant 2) \]
数乘
\[ (kA)^{-1} = k^{-1} A^{-1}, \quad \transpose{kA} = k\transpose{A}, \quad (kA)^{*} = k^{n-1} A^{*} \quad (n \geqslant 2) \]

矩阵 $A$ 可逆的等价命题
\begin{itemize}
	\item 存在 $B$ 使得 $AB = BA = E$;
	\item $|A| \neq 0$;
	\item $r(A) = n$;
	\item $A$ 可以表示为初等矩阵的乘积;
	\item $A$ 无零特征值;
	\item $A\vbf{x} = 0$ 只有零解;
\end{itemize}

\section{向量}

\subsection{大纲要求}

1. 理解 $n$ 维向量、向量的线性组合与线性表示的概念。

2. 理解向量组线性相关、线性无关的概念,掌握向量组线性相关、线性无关的有关性质及判别法。

3. 理解向量组的极大线性无关组和向量组的秩的概念,会求向量组的极大线性无关组及秩。

4. 理解向量组等价的概念,理解矩阵的秩与其行(列)向量组的秩之间的关系。

5. 了解 $n$ 维向量空间、子空间、基底、维数、坐标等概念。

6. 了解基变换和坐标变换公式,会求过渡矩阵。

7. 了解内积的概念,掌握线性无关向量组正交规范化的施密特(Schmidt)方法。

8. 了解规范正交基、正交矩阵的概念以及它们的性质。

\subsection{向量空间}

设 $\seq{\vbf{a}}{n}$ 为 $\mathbb{R}$ 中的一个基,取新基 $\seq{\vbf{b}}{n}$,则基变换公式为
\[ (\seq{\vbf{b}}{n}) =  (\seq{\vbf{a}}{n})P \]
其中 $P = A^{-1} B$ 称为旧基到新基的过渡矩阵,旧坐标 $\vbf{y}$ 到新坐标 $\vbf{z}$ 的变换公式是 $\vbf{z} = P^{-1} \vbf{y}$。

\begin{theorem}[Schmidt 正交化]
	设 $\seq{\alpha}{s}$ 是欧几里得空间 $\mathbb{R}^n$ 中一个线性无关的向量组,令
	\[
		\begin{aligned}
			\vbf{\beta}_1 & =\alpha_1                                                                                              \\
			\vbf{\beta}_2 & =\alpha_2 - \frac{(\alpha_2,\vbf{\beta}_1)}{(\vbf{\beta}_1,\vbf{\beta}_1)}\vbf{\beta}_1                \\
			              & \cdots                                                                                                 \\
			\vbf{\beta}_s & = \alpha_s-\sum_{j=1}^{s-1}\frac{(\alpha_s,\vbf{\beta}_j)}{(\vbf{\beta}_j,\vbf{\beta}_j)}\vbf{\beta}_j
		\end{aligned}
	\]
	则 $\seq{\vbf{\beta}}{s}$ 是正交向量组,并且与 $\seq{\alpha}{s}$ 等价。
\end{theorem}

\section{线性方程组}

\subsection{大纲要求}

1. 会用克拉默法则。

2. 理解齐次线性方程组有非零解的充分必要条件及非齐次线性方程组有解的充分必要条件。

3. 理解齐次线性方程组的基础解系、通解及解空间的概念,掌握齐次线性方程组的基础解系和通解的求法。

4. 理解非齐次线性方程组解的结构及通解的概念。

5. 掌握用初等行变换求解线性方程组的方法。

\subsection{线性方程组的解}

初等行变换不改变矩阵的解!!!

线性方程组
\[ \vbf{\alpha}_1x_1 + \cdots + \vbf{\alpha}_nx_n = A \vbf{x} = \vbf{\beta} \]
解的情况:
\begin{itemize}
	\item 无解:$r(A) < r(\widetilde{A})$;
	\item 有唯一解:$r(A) = r(\widetilde{A})$;
	\item 有无穷多组解:$r(A) = r(\widetilde{A}) < n$;
\end{itemize}


\begin{theorem}[Cramer 法则]
	方程组 $Ax = b$ 有唯一解的充要条件是 $|A| \neq 0$。定义 $A_j$ 为矩阵 $A$ 第 $j$ 列换为常数项后的矩阵,此时解为
	\[ \vbf{x} = \left( \frac{|A_1|}{|A|}, \cdots, \frac{|A_n|}{|A|} \right) \]
\end{theorem}

求解线性方程组:$Ax = b$;
\begin{enumerate}
	\item 利用初等行变换将增广矩阵 $\widetilde{A}$ 化为行阶梯型矩阵 $B$;
	\item 如果 $r(A) < r(\widetilde{A})$ 则方程矛盾,无解;
	\item 如果 $r(A) = r(\widetilde{A})$,进一步把 $\widetilde{A}$ 化为行最简型矩阵。
	      \begin{enumerate}
		      \item 如果 $r(A) = r(\widetilde{A}) = n$,说明方程组有唯一解。
		      \item 如果 $r(A) = r(\widetilde{A}) = r < n$,说明方程组有无穷组解,令 $r$ 个首个非零元对应的 $x$ 取做非自由未知数,其余 $n-r$ 个未知数去做自由元,令自由未知数分别等于 $\seq{c}{n-r}$,即可写出 $n-r$ 个参数的通解。
	      \end{enumerate}
\end{enumerate}

\section{矩阵的特征值和特征向量}

\subsection{大纲要求}

1. 理解矩阵的特征值和特征向量的概念及性质,会求矩阵的特征值和特征向量。

2. 理解相似矩阵的概念、性质及矩阵可相似对角化的充分必要条件,掌握将矩阵化为相似对角矩阵的方法。

3. 掌握实对称矩阵的特征值和特征向量的性质。

\subsection{特征值}

\begin{definition}[特征值,特征向量]
	设 $A$ 是 $n$ 级矩阵,如果存在数 $\lambda$ 和非零列向量 $\vbf{\alpha}$ 使得 $A \vbf{\alpha} = \lambda\vbf{\alpha}$ 那么称 $\lambda$ 是 $A$ 的一个特征值,称 $\vbf{\alpha}$ 是 $A$ 的属于特征值 $\lambda$ 的一个特征向量,称 $f(\lambda) = |\lambda E - A|$ 是 $A$ 的特征多项式。
\end{definition}

\begin{note}
	注意零向量不是特征向量!
\end{note}

如果 $\lambda_0$ 是 $A$ 的特征值,对应的 $\alpha_0$ 是 $A$ 的特征向量,则
\begin{itemize}
	\item $\lambda_0$ 是特征多项式 $f(\lambda_0) = 0$ 的一个根;
	\item $\alpha_0$ 是齐次线性方程组 $(\lambda_0 E - A) \vbf{X} = \vbf{0}$ 的一个解,该解空间的全部非零向量都是 $\lambda_0$ 的特征向量,称为特征子空间。
	\item 对于多项式 $\varphi(x)$,则 $\varphi(\lambda_0)$ 是 $\varphi(A)$ 的特征值,$\alpha_0$ 仍是其对应的特征向量。
	\item 不同特征值对应的特征向量线性无关。
	\item 若 $\lambda_0$ 是 $f(\lambda)$ 的 $k$ 重根,则称 $\lambda_0$ 的代数重数为 $k$,简称为重数。
	\item 若特征子空间的维度是 $k$,则称 $\lambda_0$ 的几何重数为 $k$。几何重数一定不超过代数重数。
\end{itemize}

如果 $n$ 级矩阵有 $n$ 个特征值,则有如下结论:
\begin{itemize}
	\item $\lambda_1 + \cdots \lambda_n = \operatorname{tr}(A)$;
	\item $\lambda_1 \cdots \lambda_n = |A|$;
\end{itemize}

\subsection{相似矩阵}

相似矩阵:如果存在 $P$ 使得 $P^{-1} A P = B$,则称 $A$ 与 $B$ 相似,$P$ 称为相似变换阵。

相似矩阵的性质:
\begin{itemize}
	\item 相似矩阵的特征多项式相同,从而特征值相同;
	\item 相似矩阵与对角矩阵 $\operatorname{diag}\{\seq{\lambda}{n}\}$ 相似,记作相似标准型;
	\item 相似矩阵的迹、行列式、秩相等;
\end{itemize}

$n$ 阶矩阵 $A$ 可对角化的充要条件:$A$ 有 $n$ 个线性无关的特征向量。此时每个特征值的几何重数都和代数重数相等。

\subsection{实对称矩阵}

实对称矩阵矩阵的性质:
\begin{itemize}
	\item 不同特征值对应的特征向量相互正交;
	\item 必能相似对角化,且可使用正交矩阵对角化;
	\item 若 $n$ 阶,则 $0$ 必为 $A$ 的 $n-r$ 重特征值;
\end{itemize}

对于 $n$ 级实对称矩阵 $A$,找一个正交矩阵 $T$,使得 $T^{-1}AT$ 为对角矩阵的步骤如下。

1. 计算 $|\lambda E- A|$,求出它的全部不同的根:$\seq{\lambda}{m}$,它们是 $A$ 的特征值。

2. 对于每一个特征值 $\lambda_j$,求 $(\lambda_jE-A)\vbf{X} = \vbf{0}$ 的一个基础解系 $\vbf{\alpha}_{j1},\cdots,\vbf{\alpha}_{jr_j}$;然后把它们 Schmidt 正交化和单位化,得到 $\vbf{\eta}_{j1},\cdots,\vbf{\eta}_{jr_j}$。它们也是 $A$ 的属于 $\lambda_j$ 的一个特征向量。

3. 令
\[ T=(\vbf{\eta}_{11},\cdots,\vbf{\eta}_{1r_1},\cdots,\vbf{\eta}_{m1},\cdots,\vbf{\eta}_{mr_m}) \]
则 $T$ 是 $n$ 级正交矩阵,且
\[ T^{-1}AT = \operatorname{diag}\{\lambda_{1},\cdots,\lambda_{1},\cdots,\lambda_{m},\cdots,\lambda_{m}\} \]

\begin{example}
	给定实矩阵 $A$,求正交矩阵 $T$ 使得 $T^{-1}AT$ 为对角矩阵
	\[ A=\left(
		\begin{matrix}
				0  & -2 & 2  \\
				-2 & -3 & 4  \\
				2  & 4  & -3 \\
			\end{matrix}
		\right) \]
\end{example}

\begin{solution}
	首先求得特征多项式
	\[ |\lambda E - A| = (\lambda-1)^2(\lambda+8) \]
	得到特征值为 $1, -8$,分别求得 $(\lambda E - A) \vbf{X} = \vbf{0}$ 的基础解系
	\[ \vbf{\alpha}_1 = \transpose{(-2, 0, 1)}, \vbf{\alpha}_2 = \transpose{(2, 1, 0)}, \vbf{\alpha}_3 = \transpose{(-1, 2, -2)} \]
	正交归一化后得到
	\[ \vbf{\eta}_1 = \transpose{\left(-\frac{2}{\sqrt{5}}, 0, \frac{1}{\sqrt{5}}\right)}, \vbf{\alpha}_2 = \transpose{\left(\frac{2}{3\sqrt{5}}, \frac{\sqrt{5}}{3}, \frac{4}{3 \sqrt{5}}\right)}, \vbf{\eta}_3 = \transpose{\left(-\frac{1}{3}, \frac{2}{3}, -\frac{2}{3}\right)} \]
	因此
	\[ T = \left(\begin{matrix}
				-\frac{2}{\sqrt{5}} & \frac{2}{3\sqrt{5}}  & -\frac{1}{3} \\
				0                   & \frac{\sqrt{5}}{3}   & \frac{2}{3}  \\
				\frac{1}{\sqrt{5}}  & \frac{4}{3 \sqrt{5}} & -\frac{2}{3}
			\end{matrix}\right) \]
\end{solution}

\section{二次型}

\subsection{大纲要求}

1. 掌握二次型及其矩阵表示,了解二次型秩的概念,了解合同变换与合同矩阵的概念,了解二次型的标准形、规范形的概念以及惯性定理。

2. 掌握用正交变换化二次型为标准形的方法,会用配方法化二次型为标准形。

3. 理解正定二次型、正定矩阵的概念,并掌握其判别法

\subsection{二次型}

\newcommand{\XAX}{\transpose{\vbf{X}}A\vbf{X}}

\begin{definition}[二次型]
	$n$ 元二次型是 $n$ 个变量的二次齐次多项式,它的一般形式是
	\[f(\seq{x}{n}) = \sum_{i=1}^n\sum_{j=1}^na_{ij}x_ix_j\]
	把二次型的系数排成一个 $n$ 级矩阵 $A$,则二次型可以写作矩阵形式
	\[f(\seq{x}{n}) = \XAX\]
\end{definition}

矩阵的合同:如果 $n$ 级矩阵 $A$ 与 $B$,如果存在可逆矩阵 $C$ 使得
\[ \transpose{C}AC = B \]
那么称 $A$ 与 $B$ 合同,记作 $A\simeq B$。

对于 $\mathbb{R}^n$ 上的向量 $\vbf{X}, \vbf{Y}$,称 $X = CY$ 是 $\vbf{X}$ 到 $\vbf{Y}$ 的线性替换,若 $C$ 是正交矩阵,则称为正交替换。替换下的二次型为
\[ \XAX = \transpose{(C\vbf{Y})} A (C\vbf{Y}) = \vbf{Y}\left(\transpose{C} A C\right)\vbf{Y} \]
替换前后的二次型等价。

二次型的等价形式
\begin{itemize}
	\item 标准型:存在可逆先行替换,使得二次型只含平方项。
	\item 规范型:系数只取 $1,-1,0$。
\end{itemize}

\begin{example}
	用正交替换把下述实二次型化为标准型
	\[ f(x, y, z) = x^2 + 2y^2 + 3z^3 - 4xy - 4yz \]
\end{example}

\begin{solution}
	首先得到矩阵
	\[ A = \left(\begin{matrix}
				1 & -2 & 0 \\ -2 & 2 & -2 \\ 0 & -2 & 3
			\end{matrix}\right) \]
	计算其特征值
	\[ |\lambda E -A| = (\lambda - 2)(\lambda - 5)(\lambda + 1) \]
	即 $A$ 的全部特征值为 $\lambda = 2, 5, -1$,对应的基础解系单位化得到
	\[ \vbf{\eta}_1 = \transpose{\left(-\frac{2}{3}, \frac{1}{3}, \frac{2}{3}\right)}, \vbf{\eta}_2 = \transpose{\left(\frac{1}{3}, -\frac{2}{3}, \frac{2}{3}\right)}, \vbf{\eta}_3 = \transpose{\left(\frac{2}{3}, \frac{2}{3}, \frac{1}{3}\right)}  \]
	令
	\[ T = \left(\begin{matrix}
				-\frac{2}{3} & \frac{1}{3}  & \frac{2}{3} \\
				\frac{1}{3}  & -\frac{2}{3} & \frac{2}{3}        \\
				\frac{2}{3}  & \frac{2}{3}  & \frac{1}{3}
			\end{matrix}\right) \]
	此处 $T$ 即是正交矩阵,且 $T^{-1}AT = \operatorname{diag}\{2, 5, -1\}$。令
	\[ \transpose{\left(x, y, z\right)} = T \transpose{\left(a, b, c\right)} \]
	则
	\[ f(x, y, z) = 2a^2 + 5y^2 - z^2 \]
\end{solution}

\subsection{正定二次型}

实数域上的二次型简称为实二次型,$n$ 元实二次型 $\XAX$ 经过一个适当的非退化线性替换 $\vbf{X} = C\vbf{Y}$ 可以化成下述形式的标准形
\[ d_1y_1^2+\cdots+d_py_p^2-d_{p+1}y_{p+1}^2-\cdots-d_ry_r^2 \]
其中 $d_i>0,i=1,\cdots,r$。再做一次非退化线性替换可以变成
\[ z_1^2+\cdots+z_p^2-z_{p+1}^2-\cdots-z_r^2 \]

定义二次型的惯性系数
\begin{itemize}
	\item 正惯性系数:规范型中系数为 $+1$ 的平方项个数;
	\item 负惯性系数:规范型中系数为 $-1$ 的平方项个数;
	\item 符号差:正惯性系数减去负惯性系数。
\end{itemize}
