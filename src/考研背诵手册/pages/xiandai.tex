\chapter{线性代数}

\section{行列式}

\subsection{大纲要求}

1. 了解行列式的概念,掌握行列式的性质。

2. 会应用行列式的性质和行列式按行(列)展开定理计算行列式。


\section{矩阵}

\subsection{大纲要求}

1. 理解矩阵的概念,了解单位矩阵、数量矩阵、对角矩阵、三角矩阵、对称矩阵和反对称矩阵以及它们的性质。

2. 掌握矩阵的线性运算、乘法、转置以及它们的运算规律,了解方阵的幂与方阵乘积的行列式的性质。

3. 理解逆矩阵的概念,掌握逆矩阵的性质以及矩阵可逆的充分必要条件,理解伴随矩阵的概念,会用伴随矩阵求逆矩阵。

4. 理解矩阵初等变换的概念,了解初等矩阵的性质和矩阵等价的概念,理解矩阵的秩的概念,掌握用初等变换求矩阵的秩和逆矩阵的方法。

5. 了解分块矩阵及其运算。


\section{向量}

\subsection{大纲要求}

1. 理解 $n$ 维向量、向量的线性组合与线性表示的概念。

2. 理解向量组线性相关、线性无关的概念,掌握向量组线性相关、线性无关的有关性质及判别法。

3. 理解向量组的极大线性无关组和向量组的秩的概念,会求向量组的极大线性无关组及秩。

4. 理解向量组等价的概念,理解矩阵的秩与其行(列)向量组的秩之间的关系。

5. 了解 $n$ 维向量空间、子空间、基底、维数、坐标等概念。

6. 了解基变换和坐标变换公式,会求过渡矩阵。

7. 了解内积的概念,掌握线性无关向量组正交规范化的施密特(Schmidt)方法。

8. 了解规范正交基、正交矩阵的概念以及它们的性质。


\section{线性方程组}

\subsection{大纲要求}

1. 会用克拉默法则。

2. 理解齐次线性方程组有非零解的充分必要条件及非齐次线性方程组有解的充分必要条件。

3. 理解齐次线性方程组的基础解系、通解及解空间的概念,掌握齐次线性方程组的基础解系和通解的求法。

4. 理解非齐次线性方程组解的结构及通解的概念。

5. 掌握用初等行变换求解线性方程组的方法。


\section{矩阵的特征值和特征向量}

\subsection{大纲要求}

1. 理解矩阵的特征值和特征向量的概念及性质,会求矩阵的特征值和特征向量。

2. 理解相似矩阵的概念、性质及矩阵可相似对角化的充分必要条件,掌握将矩阵化为相似对角矩阵的方法。

3. 掌握实对称矩阵的特征值和特征向量的性质。


\section{二次型}

\subsection{大纲要求}

1. 掌握二次型及其矩阵表示,了解二次型秩的概念,了解合同变换与合同矩阵的概念,了解二次型的标准形、规范形的概念以及惯性定理。

2. 掌握用正交变换化二次型为标准形的方法,会用配方法化二次型为标准形。

3. 理解正定二次型、正定矩阵的概念,并掌握其判别法
