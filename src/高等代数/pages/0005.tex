\chapter{矩阵的相抵与相似}

\section{等价关系与集合的划分}

等价关系还是记录一下。

\begin{definition}
	设 $S$ 是一个非空集合,我们把 $S\times S$ 的一个子集 $W$ 叫做 $S$ 上的一个二元关系。如果 $(a,b)\in W$,那么称 $a$ 与 $b$ 有 $W$ 关系;如果 $(a,b)\notin W$,那么称 $a$ 与 $b$ 没有 $W$ 关系。
\end{definition}

当 $a$ 与 $b$ 有 $W$ 关系时,记作 $aWb$,或 $a\sim b$。

\begin{definition}
	集合 $S$ 上的一个二元关系 $\sim$ 如果具有下述性质:$\forall a,b,c\in S$,有
	
	(1) 反身性 $a\sim a$
	
	(2) 对称性 $a\sim b \Rightarrow b\sim a$
	
	(3) 传递性 $a\sim b$ 且 $b\sim c \Rightarrow a\sim c$
	
	那么称 $\sim$ 是 $S$ 上的一个等价关系。
\end{definition}

\begin{definition}
	设 $\sim$ 是集合 $S$ 上的一个等价关系,$a\in S$ ,令
	\[\overline{a} \coloneqq  \{x\in S \mid x\in a\}\]
	称 $\overline{a}$ 是由 $a$ 确定的等价类,$a$ 称为等价类 $\overline{a}$ 的一个代表。
\end{definition}

\begin{definition}
	如果集合 $S$ 是一些非空子集 $S_i$ ($i\in I$,这里 $I$ 表示指标集)的并集,并且其中不相等的子集一定不相交,那么称集合 $\{S
		_i \mid i\in I\}$ 是 $S$ 的一个划分,记作 $\pi(S)$。
\end{definition}

\begin{theorem}
	设 $\sim$ 是集合 $S$ 上的一个等价关系,则所有等价类组成的集合是 $S$ 的一个划分,记作 $\pi_\sim(S)$。
\end{theorem}

\begin{definition}
	设 $\sim$ 是集合 $S$ 上的一个等价关系。由所有等价类组成的集合称为 $S$ 对于关系 $\sim$ 的商集,记作 $S/\sim$。
\end{definition}

注意,$S$ 的商集 $S/\sim$ 里的元素是 $S$ 的子集,不是 $S$ 的元素。

设 $\sim$ 是集合 $S$ 上的一个等价关系,一种量或一种表达式如果对于同一个等价类里的元素是相等的,那么称这种量或表达式是一个不变量。恰好能完全决定等价类的一组不变量称为完全不变量。

\section{矩阵的相抵}

数域 $K$ 上所有 $s\times n$ 矩阵组成的集合记作 $M_{s\times n}(K)$,当 $s=n$ 时简记为 $M_n(K)$。

\begin{definition}
	对于数域 $K$ 上的 $s\times n$ 矩阵 $A$ 和 $B$,如果从 $A$ 经过一系列初等行变换和初等列变换能变成矩阵 $B$,那么称 $A$ 与 $B$ 是相抵的,记作 $A\overset{\text{相抵}}{\sim}B$。
\end{definition}

相抵是集合 $M_{s\times n}(K)$ 上的一个二元关系,不难验证相抵是一个等价关系,其下矩阵 $A$ 的等价类称为 $A$ 的相抵类。

\begin{theorem}
	设数域 $K$ 上 $s\times n$ 矩阵 $A$ 的秩为 $r$,如果 $r>0$,那么 $A$ 相抵于下述形式的矩阵
	\[\left(\begin{matrix}
				E_r & 0  \\
				0   & 0
			\end{matrix}\right)\]
	称其为 $A$ 的相抵标准形;如果 $r=0$,那么相抵标准形是零矩阵。
\end{theorem}

\begin{theorem}
	数域 $K$ 上 $s\times n$ 矩阵 $A$ 与 $B$ 相抵当且仅当它们的秩相等,即矩阵的秩是相抵关系下的完全不变量。
\end{theorem}

\section{广义逆矩阵}

对于线性方程组 $A\vbf{X} = \vbf{\beta}$,当 $A$ 可逆时答案可以被直接表示为 $\vbf{X} = A^{-1}\vbf{\beta}$。当 $A$ 不可逆但方程有解时,引入广义逆的概念来辅助分析。

\begin{theorem}
	设 $A$ 是数域 $K$ 上 $s\times n$ 非零矩阵,则矩阵方程
	\[ A \vbf{X} A = A \]
	一定有解。如果 $\rank(A) = r$,并且
	\[ A = P\left(\begin{matrix}
				E_r & 0 \\ 0   & 0
			\end{matrix}\right)Q \]
	其中 $P,Q$ 分别是 $K$ 上 $s$ 级、$n$ 级可逆矩阵,那么矩阵方程的通解为
	\[ \vbf{X} = Q^{-1}\left(\begin{matrix}
				E_r & B \\ C   & D
			\end{matrix}\right)P^{-1} \]
	其中 $B,C,D$ 分别是数域 $K$ 上任意的 $r\times (s-r),(n-r)\times r,(n-r)\times (s-r)$ 矩阵。
\end{theorem}

\begin{definition}[广义逆] \index{guangyini@广义逆}
	设 $A$ 是数域 $K$ 上 $s\times n$ 非零矩阵,则矩阵方程 $A\vbf{X}A = A$ 的每一个解都称为 $A$ 的一个广义逆矩阵,简称广义逆。用 $A^-$ 表示。
\end{definition}

\begin{theorem}
	非齐次线性方程组 $A\vbf{X} = \vbf{\beta}$ 有解的充分必要条件是
	\[\vbf{\beta} = AA^-\vbf{\beta} \]
\end{theorem}

\begin{proof}
	必要性:设 $A \vbf{X}= \vbf{\beta}$ 有解 $\alpha$,则
	\[ \vbf{\beta} = A \vbf{\alpha} = A A^{-} A \vbf{\alpha} = A A^{-} \vbf{\beta}  \]
	充分性:若 $\vbf{\beta} = A A^{-} \vbf{\beta}$,显然 $A^{-} \vbf{\beta}$ 是 $A\vbf{X} = \vbf{\beta}$ 的解。
\end{proof}

\begin{theorem}
	非齐次线性方程组 $A\vbf{X} = \vbf{\beta}$ 有解时,它的通解为
	\[ \vbf{X}  = A^-\vbf{\beta}  \]
\end{theorem}

\begin{theorem}
	数域 $K$ 上 $n$ 元齐次线性方程组 $A\vbf{X} = 0$ 的通解为
	\[ \vbf{X}  = (E_n - A^-A)\vbf{Z} \]
	其中 $A^-$ 是 $A$ 的任意一个广义逆,$\vbf{Z}$ 取遍 $K^n$ 中任意列向量。
\end{theorem}

\begin{theorem}
	数域 $K$ 上 $n$ 元非齐次线性方程组 $A\vbf{X} = \vbf{\beta}$ 的通解为
	\[ \vbf{X}  = A^{-}\vbf{\beta} + (E_n - A^-A)\vbf{Z} \]
	其中 $A^-$ 是 $A$ 的任意一个广义逆,$\vbf{Z}$ 取遍 $K^n$ 中任意列向量。
\end{theorem}

% \begin{definition}
%	 设 $A$ 是复数域上 $s\times n$ 矩阵,矩阵方程组
%	 \[ \begin{cases}
%		 AXA = A,\\
%		 XAX = X,\\
%		 (AX)^* = AX,\\
%		 (XA)^* = XA,
%	 \end{cases} \]
%	 称为 $A$ 的 Penrose 方程组,它的解称为 $A$ 的 Moore-Penrose 广义逆,记作 $A^+$。
% \end{definition}44

\section{矩阵的相似}

\begin{definition}[矩阵相似] \index{juzhenxiangsi@矩阵相似}
	设 $A$ 与 $B$ 都是数域 $K$ 上 $n$ 级矩阵,如果存在数域 $K$ 上一个 $n$ 级可逆矩阵 $P$,使得
	\[ P^{-1}AP = B \]
	那么称 $A$ 与 $B$ 是相似的,记作 $A\sim B$。
\end{definition}

同样相似是集合 $M_{s\times n}(K)$ 上的一个二元关系,不难验证相似是一个等价关系,其下矩阵 $A$ 的等价类称为 $A$ 的相似类。相似的矩阵具有相等的行列式和秩。

\begin{definition}
	$n$ 级矩阵 $A=(a_{ij})$ 的主对角线上元素的称为 $A$ 的迹,记作 $\tr(A)$。
\end{definition}

不难验证矩阵的迹都有如下性质
\[ 
	\begin{aligned}
		\tr(A+B) & = \tr(A) + \tr(B) \\
		\tr(kA)  & = k\tr(A)         \\
		\tr(AB)  & = \tr(BA)
	\end{aligned}
\]

\begin{theorem}
	相似的矩阵都有相同的迹。
\end{theorem}

\begin{proof}
	不妨设 $A = P^{-1} B P$,于是
	\[ \tr(B) = \tr(P^{-1} AP) = \tr((P^{-1}P)A) = \tr(A) \]
\end{proof}

这表明,矩阵的行列式、秩、迹都是相似关系下的不变量,简称为相似不变量。如果 $n$ 级矩阵相似于一个对角矩阵,那么称 $A$ \textbf{可对角化} \index{keduijiaohua@可对角化}。

\begin{theorem} 
	数域 $K$ 上 $n$ 级矩阵 $A$ 可对角化的充分必要条件是,$K^n$ 中有 $n$ 个线性无关的列向量 $\seq{\vbf{\alpha}}{n}$,以及 $K$ 中有 $n$ 个数 $\seq{\lambda}{n}$(它们之中有些可能相等),使得
	\[ A\vbf{\alpha}_i = \lambda_i \vbf{\alpha}_i, \quad  1 \leqslant i \leqslant n \]
	这时,令 $P = (\seq{\vbf{\alpha}}{n})$,则
	\[ P^{-1}AP = \diag\{\seq{\lambda}{n}\} \]
\end{theorem}

\begin{proof}
	带入 $AP = PD$,注意到
	\[ AP = (\seq{A\vbf{\alpha}}{n}) = (\lambda_1 \vbf{\alpha}_1, \cdots, \lambda_n \vbf{\alpha}_n)D = PD \]
	因此存在 $A \vbf{\alpha}_i = \lambda_i \vbf{\alpha}_i$。
\end{proof}

\begin{example}
	如果 $n$ 级矩阵 $A$ 可对角化,求证:$A$ 相似于 $\transpose{A}$。
\end{example}

\begin{proof}
	即存在 $P$ 使得 $P^{-1}AP = D$,其中 $D$ 是对角阵,从而存在
	\[ \transpose{D} = \transpose{(P^{-1}AP)} = \transpose{P}\transpose{A} (\transpose{P}) ^{-1}  \]
	因此 $A \sim D = \transpose{D} ~ \transpose{A}$。
\end{proof}

\begin{example}
	如果 $n$ 级矩阵 $A, B$ 满足 $AB - BA = A$,求证:$A$ 不可逆。
\end{example}

\begin{proof}
	假设 $A$ 可逆,两边乘 $A^{-1}$ 得到
	\[ B - A^{-1} B A = E \]
	从左边来看,有
	\[ \tr(B - A^{-1} B A) = \tr(B) - \tr(A^{-1}BA) = \tr(B) - \tr(B) = 0 \]
	因此 $0 \neq \tr(E) = n$ 矛盾,故不可逆。
\end{proof}

\section{矩阵的特征值和特征向量}

\begin{definition} \index{tezhengzhi@特征值} \index{tezhengxiangliang@特征向量}
	设 $A$ 是数域 $K$ 上的一个 $n$ 级矩阵,如果 $K^n$ 中有非零列向量 $\vbf{\alpha}$ 使得
	\[ A \vbf{\alpha} = \lambda_0\vbf{\alpha},\text{且}\ \lambda_0\in K \]
	那么称 $\lambda_0$ 是 $A$ 的一个特征值,称 $\vbf{\alpha}$ 是 $A$ 的属于特征值 $\lambda_0$ 的一个特征向量。
\end{definition}

注意零向量不是特征向量。

\begin{theorem}
	设 $A$ 是数域 $K$ 上的 $n$ 级矩阵,则
	
	(1) $\lambda_0$ 是 $A$ 的一个特征值当且仅当 $\lambda_0$ 是 $A$ 的特征多项式 $|\lambda E-A|$ 在 $K$ 中的一个根。\index{tezhengduoxiangshi@特征多项式}
	
	(2) $\vbf{\alpha}$ 是 $A$ 的属于特征值 $\lambda_0$ 的一个特征向量当且仅当 $\vbf{\alpha}$ 是齐次线性方程组 $(\lambda_0E-A)\vbf{X} = \vbf{0}$  的一个解。
\end{theorem}

设 $\lambda_j$ 是 $A$ 的一个特征值,把齐次线性方程组 $(\lambda_jE-A)\vbf{X} = \vbf{0}$ 的解空间称为 $A$ 的属于 $\lambda_j$ 的特征子空间,其中的全部非零向量都是 $A$ 的属于 $\lambda_j$ 的全部特征向量。

\begin{theorem}
	相似的矩阵有相等的特征多项式。
\end{theorem}

因此矩阵的特征多项式和特征值都是相似不变量。

\begin{theorem}
	设 $A$ 是 $n$ 级矩阵,则 $A$ 的特征多项式 $f(\lambda) = |\lambda E - A|$ 是一个 $n$ 次多项式,且 $[\lambda^n]f = 1, [\lambda^{n-1}]f = -\tr(A), [\lambda^0] f = (-1)^n |A|$。
\end{theorem}

\begin{definition}
	设 $A$ 是数域 $K$ 上的 $n$ 级矩阵,$\lambda_1$ 是 $A$ 的一个特征值。把 $A$ 的属于 $\lambda_1$ 的特征子空间的维数叫做特征值 $\lambda_1$ 的几何重数,而把 $\lambda_1$ 作为 $A$ 的特征多项式根的重数叫做 $\lambda_1$ 的代数重数。代数重数简称为重数。\index{daishuchongshu@代数重数} \index{jihechognshu@几何重数}
\end{definition}

\begin{theorem}
	设 $\lambda_1$ 是数域 $K$ 上的 $n$ 级矩阵 $A$ 的一个特征值,则 $\lambda_1$ 的几何重数不超过它的代数重数。
\end{theorem}

\begin{example}
	设 $A$ 是一个 $n$ 级正交矩阵,证明:
	
	\begin{enumerate}
		\item 如果 $A$ 有特征值,那么它的特征值是 $1$ 或 $-1$;
		\item 如果 $|A| = -1$,那么 $-1$ 是 $A$ 的一个特征值;
		\item 如果 $|A| = 1$,且 $n$ 是奇数,那么 $1$ 是 $A$ 的一个特征值。
	\end{enumerate}
\end{example}

\begin{proof}
	(1)存在特征值即存在 $\lambda$ 使得 $A \vbf{\alpha} = \lambda \vbf{\alpha}$,得到
	\[ \transpose{\vbf{\alpha}}\vbf{\alpha} = \transpose{(A \vbf{\alpha})}(A \vbf{\alpha}) = \transpose{(\lambda \vbf{\alpha})}(\lambda \vbf{\alpha}) = \lambda^2  \transpose{\vbf{\alpha}}\vbf{\alpha} \]
	因此 $\lambda = \pm 1$。
	
	(2)第四章证明过。
	
	(3)如果 $|A| = 1$,那么
	\[ |E - A| = |A \transpose{A} - A| = (-1)^n |E - A| = - |E-A| \]
	因此 $1$ 是其特征值。
\end{proof}

\begin{example}
	求矩阵的全部特征值和特征向量 \[ A=\left(
		\begin{matrix}
				2  & 2  & -2 \\
				2  & 5  & -4 \\
				-2 & -4 & 5  \\
			\end{matrix}
		\right) \]
\end{example}

\begin{solution}
	首先求出特征多项式
	\[ |\lambda E - A| = \lambda^3 - 12 \lambda^2 + 21 - 10 = (\lambda - 1)^2(\lambda - 10) \]
	解得 $\lambda_1 = 1, \lambda_2 = 10$。首先解方程 $(\lambda_1 E - A) \vbf{X} = \vbf{0}$,得到基本解系
	\[ \alpha_1 = \transpose{(-2, 1, 0)}, \alpha_2 = \transpose{(2, 0, 1)}, \alpha_3 = \transpose{(-1, -2, 2)} \]
\end{solution}

\section{矩阵可对角化的条件}

\begin{theorem}
	数域 $K$ 上 $n$ 级矩阵 $A$ 可对角化的充分必要条件是 $A$ 有 $n$ 个线性无关的特征向量 $\seq{\vbf{\alpha}}{n}$,此时令
	\[ P=(\seq{\vbf{\alpha}}{n}) \]
	则
	\[ P^{-1}AP = \diag\{\seq{\lambda}{n}\} \]
	其中 $\lambda_i$ 是 $\vbf{\alpha}_i$ 所属的特征值。上述对角矩阵称为 $A$ 的相似标准形。
\end{theorem}

\begin{theorem}
	设 $\lambda_1,\lambda_2$ 是数域 $K$ 上 $n$ 级矩阵 $A$ 的不同特征值,$\seq{\vbf{\alpha}}{s}$ 与 $\seq{\vbf{\beta}}{r}$ 分别是 $A$ 的属于 $\lambda_1,\lambda_2$ 的线性无关的特征向量,则 $\seq{\vbf{\alpha}}{s},\seq{\vbf{\beta}}{r}$ 线性无关。
\end{theorem}

\begin{theorem}
	设 $\seq{\lambda}{m}$ 是数域 $K$ 上 $n$ 级矩阵 $A$ 的不同特征值,$\vbf{\alpha}_{j1},\cdots,\vbf{\alpha}_{jr_j}$ 是 $A$ 的属于 $\lambda_j$ 的线性无关的特征向量,$j=1,\cdots,m$,则向量组
	\[ \vbf{\alpha}_{11},\cdots,\vbf{\alpha}_{1r_1},\cdots,\vbf{\alpha}_{m1},\cdots,\vbf{\alpha}_{1r_m} \]
	线性无关。
\end{theorem}

\begin{theorem}
	数域 $K$ 上 $n$ 级矩阵 $A$ 可对角化的充分必要条件是:$A$ 的特征多项式的全部复根都属于 $K$,并且 $A$ 的每个特征值的几何重数等于它的代数重数。
\end{theorem}

\section{实对称矩阵的对角化}

若对于 $n$ 级矩阵 $A,B$,存在一个 $n$ 级正交矩阵 $T$,使得 $T^{-1}AT=B$,那么称 $A$ 正交相似于 $B$。

\begin{theorem}
	实对称矩阵的特征多项式的每一个复根都是实数,从而它们都是特征值。
\end{theorem}

\begin{theorem}
	实对称矩阵 $A$ 的属于不同特征值的特征向量是正交的。
\end{theorem}

\begin{theorem}
	实对称矩阵一定正交相似于对角矩阵。
\end{theorem}

对于 $n$ 级实对称矩阵 $A$,找一个正交矩阵 $T$,使得 $T^{-1}AT$ 为对角矩阵的步骤如下。

1. 计算 $|\lambda E- A|$,求出它的全部不同的根:$\seq{\lambda}{m}$,它们是 $A$ 的特征值。

2. 对于每一个特征值 $\lambda_j$,求 $(\lambda_jE-A)\vbf{X} = \vbf{0}$ 的一个基础解系 $\vbf{\alpha}_{j1},\cdots,\vbf{\alpha}_{jr_j}$;然后把它们 Schmidt 正交化和单位化,得到 $\vbf{\eta}_{j1},\cdots,\vbf{\eta}_{jr_j}$。它们也是 $A$ 的属于 $\lambda_j$ 的一个特征向量。

3. 令
\[ T=(\vbf{\eta}_{11},\cdots,\vbf{\eta}_{1r_1},\cdots,\vbf{\eta}_{m1},\cdots,\vbf{\eta}_{mr_m}) \]
则 $T$ 是 $n$ 级正交矩阵,且
\[ T^{-1}AT = \diag\{\lambda_{1},\cdots,\lambda_{1},\cdots,\lambda_{m},\cdots,\lambda_{m}\} \]

\begin{example}
	给定实矩阵 $A$,求正交矩阵 $T$ 使得 $T^{-1}AT$ 为对角矩阵
	\[ A=\left(
		\begin{matrix}
				0  & -2 & 2  \\
				-2 & -3 & 4  \\
				2  & 4  & -3 \\
			\end{matrix}
		\right) \]
\end{example}

\begin{solution}
	首先求得特征多项式
	\[ |\lambda E - A| = (\lambda-1)^2(\lambda+8) \]
	得到特征值为 $1, -8$,分别求得 $(\lambda E - A) \vbf{X} = \vbf{0}$ 的基础解系
	\[ \vbf{\alpha}_1 = \transpose{(-2, 0, 1)}, \vbf{\alpha}_2 = \transpose{(2, 1, 0)}, \vbf{\alpha}_3 = \transpose{(-1, 2, -2)} \]
	正交归一化后得到
	\[ \vbf{\eta}_1 = \transpose{\left(-\frac{2}{\sqrt{5}}, 0, \frac{1}{\sqrt{5}}\right)}, \vbf{\alpha}_2 = \transpose{\left(\frac{2}{3\sqrt{5}}, \frac{\sqrt{5}}{3}, \frac{4}{3 \sqrt{5}}\right)}, \vbf{\eta}_3 = \transpose{\left(-\frac{1}{3}, \frac{2}{3}, -\frac{2}{3}\right)} \]
	因此
	\[ T = \left(\begin{matrix}
				-\frac{2}{\sqrt{5}} & \frac{2}{3\sqrt{5}}  & -\frac{1}{3} \\ 
				0                   & \frac{\sqrt{5}}{3}   & \frac{2}{3}  \\
				\frac{1}{\sqrt{5}}  & \frac{4}{3 \sqrt{5}} & -\frac{2}{3}
			\end{matrix}\right) \]
\end{solution}
