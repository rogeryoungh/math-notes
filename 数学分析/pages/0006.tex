\chapter{微分中值定理}

\section{拉格朗日 Lagrange 定理}

\begin{theorem}[罗尔 Rolle 中值定理]
    若函数 $f$ 在 $[a,b]$ 上连续,在 $(a,b)$ 中可微,且 $f(a)=f(b)$。则存在 $\xi\in(a,b)$,使得 $f'(\xi)=0$。
\end{theorem}

\begin{theorem}
    若函数 $f$ 在 $[a,b]$ 上连续,在 $(a,b)$ 中可微,则存在 $\xi\in(a,b)$,使得
    $$f'(\xi)=\frac{f(b)-f(a)}{b-a}$$
\end{theorem}

Lagrange 公式还有下面几种等价形式

$$f(b)-f(a) = f'(\xi)(b-a),a<\xi<b$$
$$f(b)-f(a) = f'(a+\theta(b-a))(b-a),0<\theta<1$$
$$f(a+h) - f(a) = f'(a+\theta h)h,0<\theta<1$$

\section{柯西 Cauchy 中值定理}

\begin{theorem}
    设 $f,g$ 在 $[a,b]$ 上连续,在 $(a,b)$ 中可微,且 $g'(x)\ne 0$,则存在 $\xi\in (a,b)$,使得
    $$\frac{f(b)-f(a)}{g(b)-g(a)} = \frac{f'(\xi)}{g'(\xi)}$$
\end{theorem}

\section{凹凸性}

\begin{definition}
    设 $f$ 为定义在区间 $I$ 上的函数,若对 $I$ 上当任意两点 $x_1,x_2$ 和任意实数 $\lambda\in (0,1)$ 总有
    $$f(\lambda x_1+(1-\lambda)x_2) \leqslant \lambda f(x_1)+(1-\lambda)f(x_2)$$
    则称 $f$ 为 $I$ 上的凸函数。反之,如果总有
    $$f(\lambda x_1+(1-\lambda)x_2) \geqslant \lambda f(x_1)+(1-\lambda)f(x_2)$$
    则称 $f$ 为 $I$ 上的凹函数。
\end{definition}

