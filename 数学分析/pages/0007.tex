\chapter{积分的方法与技巧}

这部分我的参考书是《积分的方法与技巧》(金玉明等)。

\section{分项积分法}

若干微分式的和或差的不定积分,等于每个微分式的各自积分的和或差。

$$\int\left( f(x)+g(x)-h(x) \right)\dd x = \int f(x)\dd x + \int g(x)\dd x - \int h(x)\dd x$$

因此多项式的积分可以简单的通过积分各个单项式得到。

如果一个分式的分母为多项式,则可把它化成最简单的分式再积分。如

$$\frac{1}{x^2-a^2} = \frac{1}{2a}\left( \frac{1}{x-a}-\frac{1}{x+a} \right)$$

这里可以通过通分后待定系数得到。于是其积分为

$$\int \frac{\dd x}{x^2-a^2} = \frac{1}{2a}\ln\left|\frac{x-a}{x+a}\right|+C$$

对于更复杂的真分式的情况,若要计算的是

$$\int \frac{mx+n}{x^2+px+q} \dd x$$

分母不一定能直接分解,但总能进行配方

$$x^2+px+q = \left(x+\frac{p}{2}\right)^2+q-\frac{p^2}{4} = t^2 \pm a^2$$

再令 $A=m,B=n-\mfrac12mp$,可得

$$\int \frac{mx+n}{x^2+px+q} \dd x = \int \frac{At+B}{t^2 \pm a^2} \dd t 
= A\int \frac{t\dd t}{t^2 \pm a^2} + B\int \frac{\dd t}{t^2 \pm a^2}$$

其中

\begin{equation*}
    \begin{aligned}
        A\int \frac{t \dd t}{t^2 \pm a^2} &= \frac{A}{2}\int \frac{\dd(t^2 \pm a^2)}{t^2 \pm a^2} = \frac{A}{2} \ln|t^2 \pm a^2| +C \\
        B\int \frac{\dd t}{t^2 + a^2} &= \frac{B}{a}\arctan\frac{t}{a}+C \\
        B\int \frac{t \dd t}{t^2 - a^2} &= \frac{B}{2a}\ln\left|\frac{t-a}{t+a}\right|+C
    \end{aligned}
\end{equation*}

因此当 $p^2<4q$ 时,可以得到

\begin{equation*}
    \begin{aligned}
        \int \frac{mx+n}{x^2+px+q} &= \frac{A}{2} \ln|t^2 + a^2| + \frac{B}{a}\arctan\frac{t}{a} + C\\
        &=\frac{m}{2}\ln|x^2+px+q| + \frac{2n-mp}{\sqrt{4q-p^2}}\arctan{\frac{2x+p}{\sqrt{4q-p^2}}} + C
    \end{aligned}
\end{equation*}

当 $p^2>4q$ 时,可以得到

\begin{equation*}
    \begin{aligned}
        \int \frac{mx+n}{x^2+px+q} &= \frac{A}{2} \ln|t^2 - a^2| + \frac{B}{2a}\ln\left|\frac{t-a}{t+a}\right|\\
        &=\frac{m}{2}\ln|x^2+px+q| + \frac{2n-mp}{2\sqrt{4q-p^2}}\ln\left| \frac{2x+p-\sqrt{p^2-4q}}{2x+p+\sqrt{p^2-4q}} \right| + C
    \end{aligned}
\end{equation*}