\chapter{线性空间}

\section{域 F 上的线性空间的基与维数}

\begin{definition}
    设非空集合 $V$;设数域 $F$。在其上有向量的加法和纯量与向量的乘法。若满足以下公理

    A1 加法结合律:$(\aaa+\bbb) + \ggg = \aaa +(\bbb+\ggg)$。
    
    A2 加法交换律:$\aaa + \bbb = \bbb + \aaa$。
    
    A3 加法存在单位元 $\ling$:$\aaa + \ling = \aaa$。
    
    A4 加法逆元的存在性:对任意的 $\aaa \in V$ 总存在 $-\aaa\in V$,使得 $\aaa + (-\aaa) = \ling$。
    
    M1 乘法结合律:$\lambda(\mu \aaa) = (\lambda \mu) \aaa$。
    
    M2 乘法存在单位元 $\mathbf{1}$:$\mathbf{1} \aaa = \mathbf{1}$。
    
    D1 分配律 1:$\lambda(\aaa+\bbb) = \lambda \aaa + \lambda \bbb$。
    
    D2 分配律 2:$(\lambda + \mu)\aaa = \lambda \aaa + \mu \aaa$。

    那么称 $V$ 是域 $F$ 上的一个线性空间,$V$ 的元素称为向量,$F$ 的元素称为纯量。
\end{definition}

从 $8$ 大运算法则可以推导出线性空间 $V$ 的一些简单性质。

\begin{theorem}
    线性空间 $V$ 中零元素是唯一的。
\end{theorem}

对于 $V$ 中的一组向量,$\aaa_1,\cdots,\aaa_s$,域 $F$ 中的一组元素 $k_1,\cdots,k_s$,作纯量乘法和加法可得
\[ k_1\aaa_1+\cdots+k_s\aaa_s \]
称该向量为 $\aaa_1,\cdots,\aaa_s$ 的一个线性组合。

像 $\aaa_1,\cdots,\aaa_s$ 这样有次序的有限多个向量称为 $V$ 的一个向量组。若 $V$ 中的一个向量 $\bbb$ 可以表示成向量组 $\aaa_1,\cdots,\aaa_s$ 的一个线性组合,那么称 $\bbb$ 可以由向量组 $\aaa_1,\cdots,\aaa_s$ 线性表出。

\begin{definition}[线性相关]
    对于向量组 $\aaa_1,\cdots,\aaa_s$,若存在一组不全为 $0$ 的元素 $k_1,\cdots,k_s$ 使得
    \[ k_1\aaa_1 + \cdots + k_s\aaa_s = 0 \]
    则称该向量组线性相关。
\end{definition}

\begin{definition}
    在域 $F$ 上的线性空间 $V$ 中,向量组 $\aaa_1,\cdots,\aaa_s$ 的一个部分组称为一个极大线性无关组,若这个组本身是线性无关的,但是从这个向量组的其余向量(如果还有的话)中任取一个添加进去,得到的新的部分组都线性无关。
\end{definition}

\begin{definition}
    如果向量组 $\aaa_1,\cdots,\aaa_s$ 的每一个向量都可以由向量组 $\bbb_1,\cdots,\bbb_r$ 线性表出,那么称向量组 $\aaa_1,\cdots,\aaa_s$ 可以由向量组 $\bbb_1,\cdots,\bbb_r$ 线性表出。如果向量组 $\aaa_1,\cdots,\aaa_s$ 与 $\bbb_1,\cdots,\bbb_r$ 可以互相线性表出,则称两个向量组等价,记作
    \[ \{ \aaa_1, \cdots,\aaa_s \} \cong \{\bbb_1,\cdots,\bbb_r\}\]
\end{definition}

显然一个向量组与其任意一个极大线性无关组等价。

\begin{theorem}
    设域 $F$ 上的线性空间 $V$ 中向量组 $\bbb_1,\cdots,\bbb_r$ 可以由向量组 $\aaa_1,\cdots,\aaa_s$ 线性表出,如果 $r>s$,那么向量组 $\bbb_1\cdots,\bbb_r$ 线性相关。
\end{theorem}

容易据此推出,一个向量组的任意两个极大无关组所含的向量个数相等。

\begin{definition}
    向量组的极大线性无关组所含的向量个数称为这个向量组的秩。
\end{definition}

\begin{definition}
    设 $V$ 是域 $F$ 上的线性空间,$V$ 中的向量集 $S$ 如果满足下述两个条件:

    \num{1} 向量集 $S$ 是线性无关的。

    \num{2} $V$ 中每一个向量都可以由向量集 $S$ 线性表出。

    那么称 $S$ 是 $V$ 的一个基。当 $S$ 是有限集时,把 $S$ 的元素排序得到一个向量组,此时称这个向量组是 $V$ 的一个有序基,简称基。
\end{definition}

只含有零向量的线性空间的基为空集。

\begin{theorem}
    任意域上的任意线性空间都存在一个基。
\end{theorem}

\begin{definition}
    设 $V$ 是域 $F$ 上的线性空间,如果 $V$ 有一个基包含有限多个向量,那么称 $V$ 是有限维的;如果 $V$ 有一个基包含无穷多个向量,那么称 $V$ 是无限维的。
\end{definition}

如果域 $F$ 上的线性空间 $V$ 是有限维的,那么 $V$ 的任意两个基所含向量的个数相等。

如果域 $F$ 上的线性空间 $V$ 是无限维的,那么 $V$ 的任意一个基都包含无穷多个向量。

\begin{definition}
    设域 $F$ 上的线性空间 $V$,若 $V$ 是有限维的,那么把 $V$ 的一个基所含向量的个数称为 $V$ 的维数,记作 $\dim_{F} V$ 或简记 $\dim V$;若 $V$ 是无限维的,那么记 $\dim V = \infty$。
\end{definition}

设域 $F$ 上的线性空间 $V$,给定两个基
\[ \aaa_1,\cdots,\aaa_n; \quad \bbb_1,\cdots,\bbb_n \]
设 $V$ 中向量 $\aaa$ 在这两个基下的坐标分别为
\[ \XXX = \transpose{(x_1,\cdots,x_n)}, \YYY = \transpose{(y_1,\cdots,y_n)} \]

\section{子空间及其交与和,子空间的直和}

设域 $F$ 上的线性空间 $V$,若 $U$ 对 $V$ 下的加法和纯量乘法封闭,则称 $U$ 是 $V$ 的子空间。

容易验证 $\dim U \leqslant \dim V$。

可以构造包含向量组 $\aaa_1,\cdots,\aaa_s$ 最小的子空间:
\[ \langle \aaa_1,\cdots,\aaa_s \rangle \coloneqq \{ k_1\aaa_1+\cdots+k_s\aaa_s \mid k_i\in F \} \]
也记作 $L(\aaa_1,\cdots,\aaa_s)$,称作由 $\aaa_1,\cdots,\aaa_s$ 生成(张成)的线性子空间。

容易发现,向量组 $\aaa_1,\cdots,\aaa_s$ 的一个极大线性无关组就是 $\langle \aaa_1,\cdots,\aaa_s \rangle$ 的一个基,从而
\[ \dim \langle \aaa_1,\cdots,\aaa_s \rangle = \rank \{ \aaa_1,\cdots,\aaa_s \}\]

\begin{theorem}
    设 $V_1,V_2$ 都是域 $F$ 上线性空间 $V$ 的子空间,则 $V_1 \cap V_2$ 也是 $V$ 的子空间。
\end{theorem}

然而 $V_1 \cup V_2$ 并不是 $V$ 的一个线性子空间。构造包含 $V_1 \cup V_2$ 的一个子空间可以定义
\[ V_1+V_2 \coloneqq \{ \aaa_1+\aaa_2 \mid \aaa_1\in V_1, \aaa_2\in V_2 \} \]
容易验证,$V_1+V_2$ 是 $V$ 的子空间。

\begin{theorem}
    设域 $F$ 上的线性空间 $V$ 的有限维子空间 $V_1,V_2$,则 $V_1 \cap V_2, V_1+V_2$ 也是有限维的,且
    \[ \dim V_1 + \dim V+2 = \dim(V_1+V_2) + \dim(V_1 \cap V_2) \]
\end{theorem}

\begin{definition}
    设域 $F$ 上线性空间 $V$ 的两个子空间 $V_1,V_2$,若 $V_1+V_2$ 中每个向量都能被唯一的表示为
    \[ \aaa = \aaa_1 \aaa_2, \aaa_1 \in V_1, \aaa_2 \in V_2 \]
    则和 $V_1+V_2$ 被称为直和,记作 $V_1 \oplus V_2$。称 $V_1,V_2$ 互为补空间。 
\end{definition}

\begin{theorem}
    设域 $F$ 上线性空间 $V$ 的两个子空间 $V_1,V_2$,下列命题等价

    \num{1} 和 $V_1+V_2$ 是直和。

    \num{2} 和 $V_1+V_2$ 中零向量的表法唯一。

    \num{3} $V_1 \cap V_2 = 0$。
\end{theorem}

\begin{theorem}
    设域 $F$ 上线性空间 $V$ 的两个有限维子空间 $V_1,V_2$,下列命题等价

    \num{1} 和 $V_1+V_2$ 是直和。

    \num{4} $\dim(V_1+V_2) = \dim V_1 + \dim V_2$。

    \num{5} $V_1$ 的一个基与 $V_2$ 的一个基合起来是 $V_1+V_2$ 的一个基。
\end{theorem}

\begin{definition}
    设域 $F$ 上线性空间 $V$ 的子空间 $V_1,\cdots,V_s$,如果和 $V_1+\cdots+V_s$ 中每个向量 $\aaa$ 都能唯一的表示为
    \[ \aaa = \aaa_1 + \cdots + \aaa_s, \aaa_i \in V_i \]
    那么和 $V_1 + \cdots + V_s$ 称为直和,记作 $V_1 \oplus \cdots \oplus V_s$ 或 $\displaystyle\bigoplus_{i=1}^s V_i$。
\end{definition}

\section{域上线性空间的同构}

\begin{definition}
    设域 $F$ 上的线性空间 $V$ 与 $V'$,若存在双射 $\sigma : V \to V'$ 且保持加法与纯量加法两种运算,即对于任意 $\aaa,\bbb \in V,k\in F$ 有
    \[ \sigma(\aaa+\bbb) = \sigma(\aaa) + \sigma(\bbb), \quad \sigma(k\aaa) = k\sigma(\aaa) \]
    那么称 $\sigma$ 是 $V$ 到 $V'$ 的一个同构映射,此时称 $V$ 与 $V'$ 是同构的,记作 $V \cong V'$。
\end{definition}

\begin{theorem}
    域 $F$ 上两个有限维线性空间同构的充要条件是他们的维数相等。
\end{theorem}

\paragraph{有限域的元素个数}

设有限域 $F$ 的单位元是 $e$。若 $\char F = 0$,则 $F$ 中必然存在无穷多个元素 $e,2e,\cdots$。因此 $\char p$ 必然是一个素数。设域 $F$ 的特征为素数 $p$,令
\[ F_p = \{ 0,e,2e,\cdots,(p-1)e \} \]
容易证明 $F_p$ 对于 $F$ 的减法、乘法封闭,因此 $F_p$ 是 $F$ 的一个子环,且 $F_p$ 是一个交换环。任取 $F$ 的一个非零元 $ie$,由于 $p \nmid i$ 因此 $(i,p) = 1$。从而存在 $u,v \in \ZZ$ 使得 $ui,vp=1$。于是
\[ e = (ui+vp)e = uie + vpe = (ue)(ie) \]
设 $u = lp + r, 0 \leqslant r < p$ 则
\[ ue = (lp + r) e = lpe + re = re \in F_p \]
因此 $ie$ 在 $F_p$ 中有逆元 $re$,从而 $F_p$ 是一个域。从而 $F$ 可以看作 $F_p$ 上的线性空间,其中加法是域 $F$ 的加法,纯量加法是 $F_p$ 中元素做 $F$ 的乘法。由于 $F$ 只含有有限多个元素,因此 $F$ 也一定是有限维的。

不妨设为 $n$ 维,则 $F \cong F_p^n$。于是 $F$ 到 $F_p^n$ 有一个双射 $\sigma$,从而 $|F| = |F_p^n|$。由于
\[ F_p^n = \{ (a_1,\cdots,a_n) \mid a_i \in F_p \} \]
因此 $|F_p^n| = p^n$ 从而 $|F| = p^n$。我们得到了

\begin{theorem}
    设 $F$ 是任意有限域,则 $F$ 的元素个数是一个素数 $p$ 的方幂,其中 $p$ 是域 $F$ 的特征。
\end{theorem}

\section{商空间}

为了在线性空间 $V$ 上建立一个二元关系且使它是一个等价关系,可以先取 $V$ 的一个子空间 $W$,然后规定
\[ \aaa \sim \bbb \Leftrightarrow \aaa - \bbb \in W \]
这样就在 $V$ 上建立了一个二元关系 $\sim$,容易验证其就是等价关系。对于 $\aaa \in V$,$\aaa$ 的等价类 $\overline{\aaa}$ 为
\[ \overline{\aaa} = \{ \bbb \in V \mid \bbb \sim \aaa \} = \{ \aaa + \ggg \mid \ggg \in W \} \]
记 $\{ \aaa + \ggg \mid \ggg \in W \}$ 为 $\aaa + W$,称它为 $W$ 的一个陪集,$\aaa$ 称为这个陪集的代表。从而
\[ \aaa + W = \bbb + W \Leftrightarrow \overline{\aaa} = \overline{\bbb} \Leftrightarrow \aaa \sim \bbb \Leftrightarrow \aaa - \bbb \in W \]
因此一个陪集 $\aaa + W$ 的代表不唯一。

对于上述等价关系 $\sim$,商集 $V/\sim$ 记作 $V/W$,称它是 $V$ 对于子空间 $W$ 的商集。即
\[ V/W = \{ \aaa + W \mid \aaa \in V \} \]

尝试在 $V/W$ 中规定加法与纯量乘法运算
\begin{equation*}
    \begin{aligned}
        (\aaa + W) + (\bbb + W) &\coloneqq (\aaa + \bbb) + W\\
        k(\aaa + W) &\coloneqq k\aaa + W
    \end{aligned}
\end{equation*}
容易验证,$V/W$ 是域 $F$ 上的一个线性空间,称作 $V$ 对于 $W$ 的商空间,其中的元素是 $V$ 的一个等价类而不是向量。

\begin{theorem}
    设域 $F$ 上的一个有限维线性空间 $V$,若 $W$ 是 $V$ 的一个子空间,则
    \[ \dim(V/W) = \dim V - \dim W \]
\end{theorem}

\paragraph{余维数}

有时会遇到线性空间 $V$ 和它的子空间 $W$ 都是无限维,而商空间 $V/W$ 却是有限的情形。

\begin{definition}
    设域 $F$ 上的线性空间 $V$ 的一个子空间,若 $V$ 对 $W$ 的商空间是有限维,那么 $\dim(V/W)$ 称为子空间 $W$ 在 $V$ 中的余维数,记作 $\codim_V W$
\end{definition}

\paragraph{标准映射}

设域 $F$ 上的线性空间 $V$ 及其子空间 $W$,则有 $V$ 到商空间 $V/W$ 有一个很自然的映射
\[ \pi : \aaa \mapsto \aaa + W \]
称它为标准映射或典范映射。容易验证它是满射。

当 $W$ 不是零子空间时,$\pi$ 不是单射,商空间 $V/W$ 的一个元素 $\aaa + W$ 在 $\pi$ 下的原像集是 $W$ 的一个陪集 $\aaa + W$。这表明 $\pi$ 保持加法和纯量乘法运算。

\begin{theorem}
    域 $F$ 上线性空间 $V$ 的任意子空间 $W$ 的任意子空间 $W$ 都有补空间。
\end{theorem}


